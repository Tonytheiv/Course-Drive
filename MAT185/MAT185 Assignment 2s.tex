\documentclass[10pt]{article}

\usepackage{mathtools}
\usepackage{amsmath}
\usepackage{amssymb}
\usepackage{color}
\usepackage{fullwidth}
\usepackage{graphicx}
\usepackage[margin=0.6in]{geometry}
\usepackage{tikz}
\usepackage{float}
\usepackage[hidelinks, urlcolor=blue, linkcolor=blue, colorlinks=true]{hyperref} 
\usepackage{setspace}

\DeclarePairedDelimiterX\set[1]\lbrace\rbrace{\def\given{\;\delimsize\vert\;}#1}

\newcommand{\bcent}{\begin{center}}
\newcommand{\ecent}{\end{center}}
\newcommand{\tb}{\textbf}
\newcommand{\noin}{\noindent}
\newcommand{\benum}{\begin{enumerate}}
\newcommand{\eenum}{\end{enumerate}}
\newcommand{\bitem}{\begin{itemize}}
\newcommand{\eitem}{\end{itemize}}
\def\boxx#1{
    \framebox{
    \begin{tabular}{c}
    \\[-1pt]
    #1 \\
    \\[-1pt]
    \end{tabular}
    }
}

%%% This command makes a framed box of a chosen height.
\newcommand{\makenonemptybox}[2]{%
\par\nobreak\vspace{\ht\strutbox}\noindent
\setlength{\fboxrule}{0pt} % set this to 0pt to make invisible
\fbox{%
\parbox[c][#1][t]{\dimexpr\linewidth-2\fboxsep}{
  \hrule width \hsize height 0pt
  #2
 }%
}%
}
\makeatother    


\begin{document}

{\bcent\fontfamily{cmss}\selectfont
\begin{tabular}{c}
\textbf{}~~~~~~~~~~~~~~~~~~~~~~~~~~~~~~~~~~~~~~~~~~~~~~~~~~~~~~~~~~~~~~~~~~~~~~~~~~~~~~~~~~~~~~~\textbf{{\color{red} Due}: 11:59pm EST, Sunday February 19, 2022}\\\hline
\end{tabular}\ecent
}

{\fontfamily{cmss}\selectfont
\large\bcent\tb{}\\
\tb{}\\
\vspace{0pt}
%\tb{Term Test 1}\\

\tb{\Large MAT185 Linear Algebra}\\

\tb{Assignment 2}
\ecent}



\noin{\fontfamily{cmss}\selectfont\tb{\large Instructions:}} \\ %% Fairly standard and designed to save time; however, tweak as necessary.

\noindent Please read the {\bf MAT185 Assignment Policies \& FAQ} document for details on submission policies, collaboration rules and academic integrity, and general instructions. 

\benum


\item {\bf Submissions are only accepted by} \href{https://www.gradescope.ca}{Gradescope}. Do not send anything by email.  Late submissions are not accepted under any circumstance. Remember you can resubmit anytime before the deadline. 

\item  {\bf Submit solutions using only this template pdf}.  Your submission should be a single pdf with your full written solutions for each question. If your solution is not written using this template pdf (scanned print or digital) then your submission will not be assessed. Organize your work neatly in the space provided.  Do not submit rough work. 

\item  {\bf Show your work and justify your steps} on every question but do not include extraneous information.  Put your final answer in the box provided, if necessary.  We recommend you write draft solutions on separate pages and afterwards write your polished solutions here on this template.

\item  {\bf You must fill out and sign the academic integrity statement below}; otherwise, you will receive zero for this assignment. 


\eenum

\vspace{30pt}


\noin{\fontfamily{cmss}\selectfont\tb{\large Academic Integrity Statement:}} \\

%%% Student information

% Student 1
\fbox{
\begin{minipage}{\textwidth}
{
\vspace{0.2in}

\makebox[\textwidth]{\sffamily Full Name: Qiao Wang}

\vspace{0.2in}

\makebox[\textwidth]{\sffamily Student number: 1009027447}

\vspace{0.1in}
}
\end{minipage}
}

\vspace*{0.1in}

% Student 2
\fbox{
\begin{minipage}{\textwidth}
{
\vspace{0.2in}

\makebox[\textwidth]{\sffamily Full Name: Justin Lim}

\vspace{0.2in}

\makebox[\textwidth]{\sffamily Student number: 1008879055}

\vspace{0.1in}
}
\end{minipage}
}
~

I confirm that:

\begin{itemize} 
	\item I have read and followed the policies described in the document {\bf MAT185 Assignment Policies \& FAQ}.
	\item In particular, I have read and understand the rules for collaboration, and permitted resources on assignments as described in subsection II of the the aforementioned document. I have not violated these rules while completing and writing this assignment. 
	\item I understand the consequences of violating the University's academic integrity policies as outlined in the \href{http://www.governingcouncil.utoronto.ca/policies/behaveac.htm}{Code of Behaviour on Academic Matters}. I have not violated them while completing and writing this assignment.
\end{itemize}
By signing this document, I agree that the statements above are true. 

% You should sign this PDF after compiling. Do not write your signature using LaTeX.
\vspace{0.2in}
{\large 
\makebox[\textwidth]{\sffamily Signatures: 1)\enspace\hrulefill} 

\vspace{0.2in}

\makebox[\textwidth]{\sffamily \hspace*{20mm} 2)\enspace\hrulefill} 

}

\vfill


\pagebreak

%%% Questions

\noin{\bf 1.} Let $W$ be the subspace of ${^n}\mathbb R^n$ defined by
$$W=\{ A=[ a_{ij} ] \in {^n}\mathbb R^n \mid \sum_{j=1}^n a_{ij}=0, \mbox{ for every } i, \mbox{ and } \sum_{i=1}^n a_{ij}=0, \mbox{ for every } j\}$$


\noin What is  $\mathrm{dim}\, W$?


%Question 1

{
	\vspace*{-37pt}
	%%% Do not change the height of this box. Your work must fit inside it.
	
	\makenonemptybox{550pt}{
    \begin{spacing}{1.5}
    \begin{center}
    \begin{equation*}
    \text{Since all rows and columns sums to 0, one element is always able to be written as the negative sum of the rest.}\\
    \text{Hence, with elementary column and row operations,} \\
    \text{any }\textit{A}\in W=
    \begin{bmatrix}
    a_{11} & a_{12} & \dots & a_{1n}\\
    a_{21} & a_{22} & \dots & a_{2n}\\
    \vdots & \vdots & \ddots & \vdots\\
    a_{n1} & a_{n2} & \dots & a_{nn}
    \end{bmatrix}
    \text{can be transformed into}
    \begin{bmatrix}
    a'_{11} & a'_{12} & \dots & -\sum^{n-1}_{j=1}a'_{1j}\\
    a'_{21} & a'_{22} & \dots & -\sum^{n-1}_{j=1}a'_{2j}\\
    \vdots & \vdots & \ddots & \vdots\\
    -\sum^{n-1}_{i=1}a'_{i1} & -\sum^{n-1}_{i=1}a'_{i2} & \dots & -\sum^{n-1}_{j=1}a'_{nj}
    \end{bmatrix}.\\
    \;\\
    \text{Consider the set }\textit{S}\text{ with }\textit{k}\text{ terms},\text{ such that all matrices in the set satisfies the condition of }\textit{W}.\\
    \textit{S}=
        \left\{
        \begin{bmatrix}
            1&0&\dots&\textbf{-1}\\
            0&0&\dots&\textbf{0}\\
            \vdots&\vdots&\ddots&\textbf\vdots\\
            \textbf{-1}&\textbf0&\textbf\dots&\textbf1
        \end{bmatrix},
        \begin{bmatrix}
            0&0&\dots&\textbf0\\
            1&0&\dots&\textbf{-1}\\
            \vdots&\vdots&\ddots&\textbf\vdots\\
            \textbf{-1}&\textbf0&\textbf\dots&\textbf1
        \end{bmatrix},
        \begin{bmatrix}
            0&1&\dots&\textbf{-1}\\
            0&0&\dots&\textbf0\\
            \vdots&\vdots&\ddots&\textbf\vdots\\
            \textbf0&\textbf{-1}&\textbf{\dots}&\textbf1
        \end{bmatrix},
        \begin{bmatrix}
            0&0&\dots&\textbf0\\
            0&1&\dots&\textbf{-1}\\
            \vdots&\vdots&\ddots&\textbf\vdots\\
            \textbf0&\textbf{-1}&\textbf\dots&\textbf1
        \end{bmatrix}
        \dots\dots
        \begin{bmatrix}
            \ddots&\dots&0&\textbf0\\
            \vdots&\ddots&0&\textbf0\\
            0&0&1&\textbf{-1}\\
            \textbf0&\textbf0&\textbf{-1}&\textbf1
        \end{bmatrix}
        \right\}\\
        \;\;\;\;\;\;\;\;\;\;\;\;\;\;\;\;\;\;\;\;\;\;\;\;\;\;\;\textit{s_1}\;\;\;\;\;\;\;\;\;\;\;\;\;\;\;\;\;\;\;\;\;\;\;\;\;\;\;\;\;
        s_2\;\;\;\;\;\;\;\;\;\;\;\;\;\;\;\;\;\;\;\;\;\;\;\;\;\;\;\;\;
        s_3\;\;\;\;\;\;\;\;\;\;\;\;\;\;\;\;\;\;\;\;\;\;\;\;\;\;\;\;\;\;
        s_4\;\;\;\;\;\;\;\;\;\;\;\;\;\;\;\;\;\;\;\;\;\;\;\;\;\;\;\;\;\;\;\;\;\;\;\;
        s_k\;\;\;\;\;\;\;\;\;\;\;\;\;\;\;\;\;\;\;\;\;\;\;\;\;\;\;\;\;\;\;\;
    \end{equation*}
    \begin{spacing}{2}
    \end{spacing}
    The last row and column for each matrix is set up such that $a_{in}=-\sum^{n-1}_{j=1}a_{ij}$ and $a_{nj}=-\sum^{n-1}_{i=1}a_{ij}$.\\
    Without counting $a_{nn}$ twice, there are $(n-1)+(n-1)+1={2n-1}$ terms as such for any $A\in{^n}\mathbb{R}^n$.\\
    In the first $n-1$ rows and columns, all entries are 0 except for one 1 that is at a unique address for each matrix.
    \begin{spacing}{2.5}
    \end{spacing}
    $s_{1_{11}}=1,\;\;\;\;\;\;s_{2_{11}},\;s_{3_{11}}\dots s_{k_{11}}=0\implies \lambda_1=0\;$  if  $\;\sum_{i=1}^k\lambda_i s_i=0$\\
    $s_{2_{21}}=1,\;\;\;\;\;\;s_{1_{21}},\;s_{2_{21}}\dots s_{k_{21}}=0\implies \lambda_2=0\;$  if  $\;\sum_{i=1}^k\lambda_i s_i=0$\\
    $s_{3_{12}}=1,\;\;\;\;\;\;s_{1_{12}},\;s_{2_{12}}\dots s_{k_{12}}=0\implies \lambda_3=0\;$  if  $\;\sum_{i=1}^k\lambda_i s_i=0$\\
    $\dots\dots$\\
    Therefore, $\;\sum_{i=1}^k\lambda_i s_i=0\implies$ all $\lambda_i=0$, and \underbar{$S$ is linearly independent}.
    \begin{spacing}{2}
    \end{spacing}
    Let $c_1,\;c_2\;c_3\dots c_k=a_{11},\;a_{21}\;a_{12}\dots a_{(n-1)(n-1)}$, then any $A\in W$ can be seen as a linear combination of elements in $S$.
    \vspace{-10pt}
    \begin{align*}
    \begin{bmatrix}
    a'_{11} & a'_{12} & \dots & -\sum^{n-1}_{j=1}a'_{1j}\\
    a'_{21} & a'_{22} & \dots & -\sum^{n-1}_{j=1}a'_{2j}\\
    \vdots & \vdots & \ddots & \vdots\\
    -\sum^{n-1}_{i=1}a'_{i1} & -\sum^{n-1}_{i=1}a'_{i2} & \dots & -\sum^{n-1}_{j=1}a'_{nj}
    \end{bmatrix}
    &=
    c_1\cdot s_1+c_2\cdot s_2+\dots+c_k\cdot s_k\\&=
    \begin{bmatrix}
        c_1&c_3&\dots&-c_1-c_3-...\\
        c_2&c_4&\dots&-c_2-c_4-...\\
        \vdots&\vdots&\ddots&\vdots\\
        -c_1-c_2-...&-c_3-c_4-...&\dots&c_1+c_2+...+c_k
    \end{bmatrix}
    \end{align*}
    Therefore, \underbar{$S$ spans $W$}. It is a \underbar{basis for $W$} with $k=n^2-(2n-1)=(n-1)^2$ elements. \\
    By the definition of dimension, \underbar{\textbf{dim} $W=(n-1)^2$}.
    
    \end{center}
    \end{spacing}
} 
\pagebreak
\newpage
\noin{\bf 2.} Let $X$ be any non-empty set and define $F(X)=\{ f \mid f: X\to \mathbb R\}$.  Define vector addition and scalar multiplication in $F(X)$ in the usual way:
\begin{align*}
(f+g)(x)=f(x)+g(x)\\
(cf)(x)=cf(x),\,\,\, c\in \mathbb R
\end{align*}
for all $x\in X$.  Then $F(X)$ is a vector space (cf. Medici, Section 4.2, page 105).
\vspace{10pt}

\noin Let $n$ be a positive integer.  If $X=\{ 1, 2, 3, \dots, n \}$, what is $\mathrm{dim}\, F(X)$?

%Question 2
{
	\vspace*{-10pt}
	%%% Do not change the height of this box. Your work must fit inside it.
	
	\makenonemptybox{550pt}{
	%%% Your work goes here! 
    \begin{center}
    Any $f_i\in F(X)$ can be written as: 
    $f_i(x)$=
    \begin{cases} \phantom{-} 1\to f_i(1)\\ \phantom{-}  2\to f_i(2) \\ \phantom{-} \;\;\;\;\;\vdots\\ \phantom{-} n\to f_i(n) \end{cases}=
     % \begin{cases} \phantom{-} a_1,\text{    if } x=1\\ \phantom{-}  a_2,\text{    if } x=2 \\ \phantom{-} \;\;\vdots\;, \;\;\;\;\;\; \vdots \\ \phantom{-} a_n,\text{    if } x=n \end{cases}=
    \begin{cases} \phantom{-} 1\to a_1\\ \phantom{-}  2\to a_2 \\ \phantom{-} \;\;\;\;\;\vdots\\ \phantom{-} n\to a_n \end{cases}, where $a_1\dots a_n\in \mathbb{R}$
    $$\;$$
    Consider the following set with $n$ elements:
    $$\;$$
    $S=$$\left\{
    f_1=\begin{cases} \phantom{-} 1\to 1\\ \phantom{-}  2\to 0 \\ \phantom{-} \;\;\;\;\;\vdots\\ \phantom{-} n\to 0 \end{cases},\;\;\;\;\;\;\;
    f_2=\begin{cases} \phantom{-} 1\to 0\\ \phantom{-}  2\to 1 \\ \phantom{-} \;\;\;\;\;\vdots\\ \phantom{-} n\to 0 \end{cases}
    \dots\dots\;\;\;\;\;
    f_n=\begin{cases} \phantom{-} 1\to 0\\ \phantom{-}  2\to 0 \\ \phantom{-} \;\;\;\;\;\vdots\\ \phantom{-} n\to 1 \end{cases}
    \right\}$
    
    % proof of the set being linearly independent
    \begin{align*}
    \text{for }X=\;&1:\;\lambda_1f_1+\lambda_2f_2+\dots+\lambda_nf_n=\lambda_1\cdot1+0+\dots+0=0\implies\lambda_1=0\\ 
    \text{for }X=\;&2:\;\lambda_1f_1+\lambda_2f_2+\dots+\lambda_nf_n=0+\lambda_2\cdot1+\dots+0=0\implies\lambda_2=0\\ 
    &\vdots\\
    \text{for }X=\;&n:\;\lambda_1f_1+\lambda_2f_2+\dots+\lambda_nf_n=0+0+\dots+\lambda_n\cdot1=0\implies\lambda_n=0\\ 
    \end{align*}
    $$\text{Therefore, }\underbar{S\text{ is linearly independent}}\text{ as it satisfies the condition }\sum_{i=1}^m\lambda_if_i=0,\implies\text{all }\lambda_i=0.
    $$
    % proof that the span of S being equal to F(X)
    \begin{align*}
        \text{Let }f_\mu=&\begin{cases} \phantom{-} 1\to r_1\\ \phantom{-}  2\to r_2 \\ \phantom{-} \;\;\;\;\;\vdots\\ \phantom{-} n\to r_n \end{cases},\;r_1\dots r_n\in\mathbb{R}
    \text{ represent any arbitrary element in }F(X)\text{ and }c_1\dots c_n=r_1\dots r_n.\\
        f_\mu=c_1\cdot
    &\begin{cases} \phantom{-} 1\to 1\\ \phantom{-}  2\to 0 \\ \phantom{-} \;\;\;\;\;\vdots\\ \phantom{-} n\to 0 \end{cases}+c_2\cdot
    \begin{cases} \phantom{-} 1\to 0\\ \phantom{-}  2\to 1 \\ \phantom{-} \;\;\;\;\;\vdots\\ \phantom{-} n\to 0 \end{cases}+\;\dots\;+c_n\cdot
    \begin{cases} \phantom{-} 1\to 0\\ \phantom{-}  2\to 0 \\ \phantom{-} \;\;\;\;\;\vdots\\ \phantom{-} n\to 1 \end{cases}=
    \begin{cases} \phantom{-} 1\to c_1=r_1\\ \phantom{-}  2\to c_2=r_2 \\ \phantom{-} \;\;\;\;\;\vdots\\ \phantom{-} n\to c_n=r_n \end{cases}
    \end{align*}
    Hence, any vector in $F(X)$ can be written as a linear combination of the elements in $S$ and \underbar{$S$ spans $F(X)$}.
    $$\;$$
    Therefore, \underbar{$S$ is a basis of $F(X)$}. Since $S$ has $n$ elements, by the definition of dimension, \underbar{\textbf{dim} $F(X)=n$}.
    \end{center}

}  




\pagebreak


\noin{\bf 3.}  Let $U$ and $W$ be subspaces of a vector space $V$.  If $\mathrm{dim}\, U=k$, and $\mathrm{dim}\, W =l$, prove that $\mathrm{dim}\, (U+W) \leq k+l$.  Under what condition would $\mathrm{dim}\, (U+W) = k+l$?  \\



%Question 3
{
	\vspace*{-10pt}
	%%% Do not change the height of this box. Your work must fit inside it.
	
	\makenonemptybox{550pt}{
	
	%%% Your work goes here! 
    \begin{spacing}{1.5}
    \begin{align*}
    U+W&=\left\{u+w\big| u\in V \text{ and } w\in W\right\}\\
    % &=\sum_{i=1}^k\lambda_i u_i+\mu_i w_i\\
    &=\textbf{\text{span }}A+\textbf{\text{span }}B\text{, where \textit{A} and \textit{B} are bases for \textit{U} and \textit{W} respectively}\\
    &=\left\{\sum_{i=1}^m\lambda_ia_i+\sum_{j=1}^n\mu_ib_i\Bigg|a\in A \text{ and } b\in B,\; \lambda,\;\mu\in\mathbb{R}\right\}\\
    &=\textbf{\text{span }}\{a_1,\;a_2\dots a_k\}\cup \{b_1,\;b_2\dots b_l\}\\
    \end{align*}
    \end{spacing}
    Case 1 $\{a_1,\;a_2\dots a_k\}\cup \{b_1,\;b_2\dots b_l\}$ linearly independent:
    \begin{center}
    \begin{spacing}{1.5}
    \end{spacing}
    Since the union set spans $U+W$ and is linearly independent, it is a basis for $U+W$, with $k+l$ elements.\\
    \begin{spacing}{1.5}
    \end{spacing}
    $\implies$ \textbf{dim} $(U+W)=k+l$\\
    \end{center}
    \begin{spacing}{1.5}
    \begin{spacing}{3}
    \end{spacing}
    Case 2 $\{a_1,\;a_2\dots a_k\}\cup \{b_1,\;b_2\dots b_l\}$ linearly dependent:\\ 
    \vspace{-25pt}
    \begin{spacing}{1}
    \end{spacing}
    \begin{center}
    \begin{spacing}{1}
    \end{spacing}
    $\{a_1,\;a_2\dots a_k\}\cup \{b_1,\;b_2\dots b_l\}$ can be rewritten as $\{a_1,\;a_2\dots a_{n-m}\dots a_{n+m}\dots a_{k-m}\}\cup \{b_1,\;b_2\dots b_{c-d}\dots b_{c+d}\dots b_{l-d}\}$\\
    where $m+d$ vectors can be written as a linear combinations of other elements in the set and are removed.\\
    \begin{spacing}{3}
    \end{spacing}
    Let $n,\;m,\;c,\;d$ be non-negative integers, with $m+d$ representing the number of linearly dependent vectors.\\
    Removing the vectors that are linearly dependent, $\{a_1,\;a_2\dots a_{n-m}\dots a_{n+m}\dots a_{k-m}\}\cup \{b_1,\;b_2\dots b_{c-d}\dots b_{c+d}\dots b_{l-d}\}$ is now linearly independent and spans $U+W$, making it a basis for $U+W$, with $k+l-(m+d)$ elements.\\
    $\implies$\textbf{dim} $(U+W)=k+l-(m+n)<k+l$\\
    \begin{spacing}{3}
    \end{spacing}
    In conclusion, if the basis of $U$ union the basis of $W$ is linearly independent, \textbf{dim} $(U+W)=k+l$, but if they are linearly dependent,  \textbf{dim} $(U+W)<k+l$. \\
    \begin{spacing}{3}
    \end{spacing}
    Therefore, combining the two cases, \textbf{dim} $(U+W)\leq k+l$ as required.
    \begin{spacing}{3}
    \end{spacing}
    Therefore, \underbar{if the union of the basis of $U$ and $W$ is linearly independent}, \textbf{dim} $(U+W)=k+l.$
    \end{center}
    \end{spacing}
     \begin{spacing}{3}
     \end{spacing}
	}
}

\pagebreak

\noin{\bf 4.}  Show that if 
$$A=\sum_{i=1}^k {\bf x}_i{\bf y}_i^{\mathrm{T}}$$
for some ${\bf x}_1, {\bf x}_2, \dots, {\bf x}_k \in {^m}\mathbb R$, and ${\bf y}_1, {\bf y}_2, \dots, {\bf y}_k \in {^n}\mathbb R$, then $\mathrm{rank}\, A \leq k$.\\



%Question 4 
    
    {
	\vspace*{-10pt}
	%%% Do not change the height of this box. Your work must fit inside it.
	
	\makenonemptybox{550pt}{
	%%% Your work goes here! 
 \begin{spacing}{1.5}
        \begin{center}
        The problem can be reformulated as:\\
        \vspace{-20pt}
        \begin{align*}
        A&=\sum_{i=1}^k
        \begin{bmatrix}
            x_{i1  }\\
            x_{i2  }\\
            \vdots\\
            x_{im  }\\
        \end{bmatrix}
        \begin{bmatrix}
            y_{i1}&y_{i2}\dots&y_{in}
        \end{bmatrix}\\
        &=\sum_{i=1}^k
        \begin{bmatrix}
            y_{i1}
            \begin{bmatrix}
            x_{i1  }\\
            x_{i2  }\\
            \vdots\\
            x_{im  }\\
            \end{bmatrix}
            &
            y_{i2}
            \begin{bmatrix}
            x_{i1  }\\
            x_{i2  }\\
            \vdots\\
            x_{im  }\\
            \end{bmatrix}
            &
            \dots\dots
            &
            y_{in}
            \begin{bmatrix}
            x_{i1  }\\
            x_{i2  }\\
            \vdots\\
            x_{im  }\\
            \end{bmatrix}
        \end{bmatrix}\\
        &=
        \begin{bmatrix}
            \sum_{i=1}^k
            \left(
            y_{i1}
            \begin{bmatrix}
            x_{i1  }\\
            x_{i2  }\\
            \vdots\\
            x_{im  }\\
            \end{bmatrix}
            \right)
            &
            \sum_{i=1}^k
            \left(
            y_{i2}
            \begin{bmatrix}
            x_{i1  }\\
            x_{i2  }\\
            \vdots\\
            x_{im  }\\
            \end{bmatrix}
            \right)
            &
            \dots\dots
            &
            \sum_{i=1}^k
            \left(
            y_{in}
            \begin{bmatrix}
            x_{i1  }\\
            x_{i2  }\\
            \vdots\\
            x_{im  }\\
            \end{bmatrix}
            \right)
        \end{bmatrix}
        \end{align*}
        Therefore, every column in $A$ can be presented as a linear combination of vectors in $\{x_i\dots x_k\}$, where elements in $y^T$ are the scalar multiples.\\
        This means that \textbf{col} $A$ is all linear combinations of $\{x_i\dots x_k\}$, and that it satisfies the subspace test and is a subspace of \textbf{span} $\{x_1\dots x_k\}$.\\
        \begin{spacing}{3}
        \end{spacing}
        By \textit{Theorem VI} in chapter 6:\\
        \textbf{col} $A\subseteq$ \textbf{span} $\{x_1\dots x_k\}\implies$ \textbf{dim col} $A\leq$ \textbf{dim span}$\{x_1\dots x_k\}$\\
        \begin{spacing}{3}
        \end{spacing}
        By the definition of rank, \textbf{dim col} $A$ = \textbf{rank} $A$, hence:\\
        \end{center}
        Case 1 $\{x_1\dots x_k\}$ linearly dependent:
        \begin{center}
            \textbf{rank} $A\leq$ \textbf{dim span} $\{x_1\dots x_k\}< k$
        \end{center}
        Case 2 $\{x_1\dots x_k\}$ linearly independent:
        \begin{center}
            \textbf{rank} $A\leq$ \textbf{dim span} $\{x_1\dots x_k\}= k$
        \end{center}
        \begin{center}
            Therefore, combining the 2 cases, \textbf{rank} $A\leq k$ as required. 
        \end{center}
\end{spacing}
	}
}


\end{document}
